\documentclass[11pt,a4paper,french]{article}
\usepackage{epsfig}
\usepackage[francais]{babel}
\usepackage{indentfirst}
\usepackage{amsmath}
\usepackage{amssymb}
\frenchspacing

\setlength{\textheight}{220mm}
\setlength{\textwidth}{150mm}
\setlength{\unitlength}{1mm}
\setlength{\oddsidemargin}{0cm}
\setlength{\evensidemargin}{0mm}
\setlength{\topmargin}{0cm}
\setlength{\footskip}{1.5cm}
%\setlength{\headsep}{5mm}
\renewcommand{\baselinestretch}{1.2}

\def\ite{\item[$\bullet$]}
\def\bi{\begin{itemize}}
\def\ei{\end{itemize}}
\def\be{\begin{enumerate}}
\def\ee{\end{enumerate}}
\def\beq{\begin{equation}}
\def\eeq{\end{equation}}
\def\beqa{\begin{eqnarray}}
\def\eeqa{\end{eqnarray}}
\def\ba{\begin{array}}
\def\ea{\end{array}}
\def\bt{\begin{tabular}}
\def\et{\end{tabular}}
\def\bc{\begin{center}}
\def\ec{\end{center}}
\def\bL{\begin{Large}}
\def\eL{\end{Large}}
\def\bl{\begin{large}}
\def\el{\end{large}}
\def\bh{\begin{huge}}
\def\eh{\end{huge}}
\def\bH{\begin{Huge}}
\def\eH{\end{Huge}}
\def\k{\vec{k}}
\def\x{\vec{x}}
\def\y{\vec{y}}
\def\z{\vec{z}}
\def\X{\vec{X}}
\def\Y{\vec{Y}}
\def\Z{\vec{Z}}
\def\u{\vec{u}}
\def\v{\vec{v}}
\def\w{\vec{w}}
\def\e{\vec{e}}
\def\Vec#1{\stackrel{\longrightarrow}{#1}}

\def\gcro{\left[}
\def\dcro{\right]}
\def\gpar{\left(}
\def\dpar{\right)}
\def\arg#1{\overline{#1}}
\def\rap#1{\left.\!\!\!\! \ba{c} \\ \ea \right|_{#1}}
%*****************   derivees   *********************
\def\pb#1#2{\frac{\partial#1}{\partial#2}}
\def\pbb#1#2{\frac{\partial^2#1}{\partial#2^2}}
%*****************   operateurs courants  ***********
\def\rot#1{\vec{\bf \nabla} \wedge #1}
\def\grad{\vec{\bf  \nabla} }
\def\div{\bf  \nabla .}
%*****************   integrales   *******************
\def\dint#1{\displaystyle \int \! \! \!\int_{\cal #1}}
\def\tint#1{\displaystyle \int \! \! \int \! \! \int_{\cal #1}}
\pagestyle{empty}
\begin{document}
\bc
{\large \bf ANALYSE NUMERIQUE}

{\large \it Exercices}
\vskip 10mm
{\bf L3 Physique et Applications }

\today
\ec

%%%%%%%%%%%%%%%%%%%%%%%%%%%%%%%%%%%%%%%%%%%%%%%%%%%%%%%%%%%%%%%%%%%%%%%
\section*{Exercice 1}
On donne la matrice~:
$$
A = \left[ \begin{array}{ccc} 	1 & a & a \\
			a & 1 & a \\
			a & a & 1
\end{array} \right]
$$

\be
\item Pour quelles valeurs de $a$ la matrice $A$ est-elle {\bf sym\'etrique d\'efinie positive ($det A > 0$)} ?
\item Posons $a=\frac{1}{4}$. Faire la d\'ecomposition $LU$ de $A$.
\item Donner la matrice de {\bf Gauss-Seidel} en fonction de $a$ et \'etudier sa convergence.
\ee

\section*{Exercice 2}
On consid\`ere le syst\`eme lin\'eaire $AX=b$:
$$
A = \left[ \begin{array}{ccc} 	\alpha & \gamma & 0 \\
			\gamma & \alpha & \beta \\
			0 & \beta & \alpha
\ea \right]
$$
et $\alpha, \beta, \gamma$ sont trois param\`etres r\'eels.
\be
\item Sans construire les matrices d'it\'eration, donner les conditions suffisantes sur les param\`etres $\alpha, \beta, \gamma$ telles que les m\'ethodes de Jacobi et de Gauss-Seidel soient convergentes.
\item Soit $J$ la matrice d'it\'eration de la m\'ethode de Jacobi.
\be
\item Ecrire la matrice $J$.
\item Donner $\rho(J)$ le rayon spectral de $J$.
\item Donner les conditions n\'ecessaires sur $\alpha, \beta, \gamma$ pour que la m\'ethode soit convergente.
\item Donner les conditions sur $\alpha, \beta, \gamma$ pour que la matrice $A$ soit d\'efinie positive et montrer dans ce cas, que $\rho(G) = \rho^2(J) $ o\`u $\rho(G)$ est le rayon spectral de la matrice de Gauss-Seidel.
\ee
\ee

\end{document}
